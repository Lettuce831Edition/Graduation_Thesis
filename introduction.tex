\chapter{序論}
\section{研究背景}
\begin{itemize}
  \item 干渉関係を知りたい
\end{itemize}

\section{先行研究の課題}
\begin{itemize}
  \item 先行研究でやってたことを簡単に説明
  \item 実環境との違い
\end{itemize}

\section{研究目的}
\begin{itemize}
  \item シミュレーション方法
  \item 先行研究との違い
  \item 評価方法
  \item 最終的にどうしたいのか
\end{itemize}

\chapter{わからない}
\section{通信方式}
\begin{itemize}
  \item CSMA/CAの通信方式を簡単に説明
  \item キャリアセンスについて説明できればいい
\end{itemize}

\section{MIC}
\begin{itemize}
  \item MICの説明
\end{itemize}

\section{代表値の選定}
\begin{itemize}
  \item 履歴の記録方法
  \item 中央値と最頻値が一緒になる
\end{itemize}

\section{パスロス}
\begin{itemize}
  \item パスロスの説明(横距離、縦損失のグラフ)
  \item 正答データとして扱える理由
\end{itemize}

\section{評価方法}
\begin{itemize}
  \item パスロスを正答データとする
  \item 相関係数を求めて評価する
\end{itemize}

\chapter{シミュレーション}
\section{シミュレーション環境}
\begin{itemize}
  \item エリアの大きさや端末の配置等
  \item パラメータの表
\end{itemize}

\section{推定結果}
\begin{itemize}
  \item 代表値を平均値とした時の横区間、縦相関値のグラフ
  \item 代表値を中央値、最頻値とした時の横区間、縦相関値のグラフ
  \item もしかしたまとめるかも
\end{itemize}

\chapter{結論}
\begin{itemize}
  \item 代表値良かったもの
\end{itemize}