\chapter{序論}
\section{研究背景}
\begin{itemize}
  \item 干渉関係を知りたい
\end{itemize}

\section{先行研究の課題}
\begin{itemize}
  \item 先行研究でやってたことを簡単に説明
  \item 実環境との違い
\end{itemize}

\section{研究目的}
\begin{itemize}
  \item シミュレーション方法
  \item 先行研究との違い
  \item 評価方法
  \item 最終的にどうしたいのか
\end{itemize}

\chapter{わからない}
\section{通信方式}
\begin{itemize}
  \item CSMA/CAの通信方式を簡単に説明
  \item キャリアセンスについて説明できればいい
\end{itemize}

\section{MIC}
\begin{itemize}
  \item MICの説明
\end{itemize}

\section{代表値の選定}
\begin{itemize}
  \item 履歴の記録方法
  \item 中央値と最頻値が一緒になる(これを代表値にする手法を考案出来たら)
\end{itemize}

\begin{table}[H]
	\centering
	\caption{各データの設定}
	\begin{tabular}{|c|c|c|}\hline
      & 1 & 0 \\ \hline
    送信履歴 & 送信時 & それ以外 \\ \hline
    CS履歴 & CS時 & それ以外 \\ \hline
    受信履歴 & 受信時 & それ以外 \\ \hline
  \end{tabular}
\end{table}

\section{パスロス}
\begin{itemize}
  \item パスロスの説明(横距離、縦損失のグラフ)
  \item 正答データとして扱える理由
\end{itemize}

\section{評価方法}
\begin{itemize}
  \item パスロスを正答データとする
  \item 相関係数を求めて評価する
\end{itemize}

\chapter{シミュレーション}
\section{シミュレーション環境}
\begin{itemize}
  \item エリアの大きさや端末の配置等
  \item パラメータの表
\end{itemize}

\section{推定結果}
平均値を代表値とする手法しかできなかった場合
\begin{itemize}
  \item 代表値を平均値とした時の横区間、縦相関値のグラフ(ピアソンとMICの両方)
  \item 時系列データの散布図(MICによって相関値が上昇した)
\end{itemize}

代表値の取り方をいい感じに考えついた場合
\begin{itemize}
  \item 代表値を平均値とした時の横区間、縦相関値のグラフ(ピアソンとMICの両方)
  \item 時系列データの散布図(MICによって相関値が上昇した)
  \item 別の代表値の取り方でやったときの横区間、縦相関値のグラフ
  \item 時系列データの散布図
  \item 相関値が上がった理由
\end{itemize}

% \begin{figure}[H]
%   \includegraphics[width=7cm]{}
%   \centering
%   \caption{}
%   \label{fig:}
% \end{figure}

\chapter{結論}
\begin{itemize}
  \item いい感じにまとめる
\end{itemize}