\documentclass[11pt, dviodfmx]{jreport}

\usepackage{treatise}
\usepackage{receipt}

\usepackage{ascmac}
\usepackage{bm}
\usepackage{fancyhdr} % ヘッダー・フッター
\usepackage{amsmath}
\usepackage{wrapfig}
\usepackage{cases}
\usepackage[subrefformat=parens]{subcaption}
\usepackage{comment} % コメントアウト
\usepackage{boites}
\usepackage{framed, color}
\usepackage{enumitem, amssymb}

%しおり用
% hyperref - begin
\usepackage[   % (括弧内はデフォルト値)
dvipdfm,
draft=false,   % hyperref の機能の有効化 (false)
bookmarks=true,   % しおり (true)
bookmarksnumbered=true,   % しおりに節番号を振るか (false)
bookmarksopen=false,   % しおりのツリーを開く (false)
bookmarkstype=toc,   % しおりのタイプ
colorlinks=false,   % リンクの色付け (false)
anchorcolor=black,   % アンカーテキストの色指定 (black)
citecolor=black,   % 参考文献リンクの色 (green)
filecolor=black,   % ローカルファイルリンクの色 (magenta)
urlcolor=black,   % URL の色 (magenta)
pdfauthor={Author},   % 著者名
pdfborder={0 0 0},   % 枠 (1 0 0)
pdftitle={Title},   % タイトル
pdfsubject={Subject},   % サブタイトル
pdfkeywords={Keywords}   % キーワード
]{hyperref}
% hyperref - end

% hyperref 文字化け回避
\usepackage{pxjahyper}

\begin{document}
% 受領表
\begin{treatise}
	\受領年度{令和4年度}
	\論文題目{\huge}{スパースモデリングを用いた干渉関係推定}
	\論文題目英語{\normalsize}{Interference\,relationship\,estimation\,using\,sparsemodeling}
	\論文執筆者{5316}{小林\,\,慧悟}
	%\論文執筆者二名{学生番号}{氏名}{学生番号}{氏名}
	\指導教員{稲毛\,\,契}{准教授}
	\長名{山本\,\,哲也}{教授}
	\受領日{令和5年xx月yy日}
\end{treatise}

% 表紙
\begin{treatiseTitle}
	\年度{令和4年度}
	\日本語タイトル{\huge}{スパースモデリングを用いた干渉関係推定}
	\英語タイトル{\large}{Interference\,relationship\,estimation\,using\,sparsemodeling}
	\筆者一名{5316}{小林\,\,慧悟}
	%\筆者二名{学生番号}{氏名}{学生番号}{氏名}
	%\筆者三名{}{}{}{}{}{}
	%\筆者四名{}{}{}{}{}{}{}{}
	\指導教員一名{稲毛\,\,契}{准教授}
	\提出日{令和5年xx月yy日}
\end{treatiseTitle}

\pagestyle{fancy}%
\lhead{\rightmark}    %左側ヘッダの定義
%\chead{¥leftmark}    中央ヘッダの定義
\rhead{令和4年度\,\,稲毛研究室\,\,卒業論文}    %右側ヘッダの定義
%\lfoot[1]{奇数ページ}    左側フッターの定義
%\cfoot[1]{}    中央フッターの定義
%\rfoot[偶数ページ]{奇数ページ}    右側フッターの定義
%\renewcommand{\headrulewidth}{0.1mm}   % ヘッダの線の太さ
%\renewcommand{\footrulewidth}{3mm}    %フッターの線の太さ

\clearpage
\addcontentsline{toc}{chapter}{概要}
\begin{center}
	{\huge 概要}
\end{center}

% -------------------- %

ここに書く。ここに概要を書く。

\tableofcontents
\listoftables
\listoffigures
\clearpage

%\graphicspath{{./figures/}} % 図が特定のフォルダにある場合には設定

\chapter{序論}
\section{研究背景}
\begin{itemize}
  \item 干渉関係を知りたい
\end{itemize}

\section{先行研究の課題}
\begin{itemize}
  \item 先行研究でやってたことを簡単に説明
  \item 実環境との違い
\end{itemize}

\section{研究目的}
\begin{itemize}
  \item シミュレーション方法
  \item 先行研究との違い
  \item 評価方法
  \item 最終的にどうしたいのか
\end{itemize}

\chapter{わからない}
\section{通信方式}
\begin{itemize}
  \item CSMA/CAの通信方式を簡単に説明
  \item キャリアセンスについて説明できればいい
\end{itemize}

\section{MIC}
\begin{itemize}
  \item MICの説明
\end{itemize}

\section{代表値の選定}
\begin{itemize}
  \item 履歴の記録方法
  \item 中央値と最頻値が一緒になる
\end{itemize}

\section{パスロス}
\begin{itemize}
  \item パスロスの説明(横距離、縦損失のグラフ)
  \item 正答データとして扱える理由
\end{itemize}

\section{評価方法}
\begin{itemize}
  \item パスロスを正答データとする
  \item 相関係数を求めて評価する
\end{itemize}

\chapter{シミュレーション}
\section{シミュレーション環境}
\begin{itemize}
  \item エリアの大きさや端末の配置等
  \item パラメータの表
\end{itemize}

\section{推定結果}
\begin{itemize}
  \item 代表値を平均値とした時の横区間、縦相関値のグラフ
  \item 代表値を中央値、最頻値とした時の横区間、縦相関値のグラフ
  \item もしかしたまとめるかも
\end{itemize}

\chapter{結論}
\begin{itemize}
  \item 代表値良かったもの
\end{itemize} % 序論

\addcontentsline{toc}{chapter}{参考文献}

\end{document}
